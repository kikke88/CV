%%%%%%%%%%%%%%%%%%%%%%%%%%%%%%%%%%%%%%%%%
% "ModernCV" CV and Cover Letter
% LaTeX Template
% Version 1.11 (19/6/14)
%
% This template has been downloaded from:
% http://www.LaTeXTemplates.com
%
% Original author:
% Xavier Danaux (xdanaux@gmail.com)
%
% License:
% CC BY-NC-SA 3.0 (http://creativecommons.org/licenses/by-nc-sa/3.0/)
%
% Important note:
% This template requires the moderncv.cls and .sty files to be in the same 
% directory as this .tex file. These files provide the resume style and themes 
% used for structuring the document.
%
%%%%%%%%%%%%%%%%%%%%%%%%%%%%%%%%%%%%%%%%%

\documentclass[11pt,a4paper,sans]{moderncv} % Font sizes: 10, 11, or 12; paper sizes: a4paper, letterpaper, a5paper, legalpaper, executivepaper or landscape; font families: sans or roman


%\usepackage{fontspec}

\usepackage[utf8]{inputenc}

\usepackage[export]{adjustbox}

\usepackage{multicol}
\moderncvstyle{casual} % CV theme - options include: 'casual' (default), 'classic', 'oldstyle' and 'banking'
\moderncvcolor{blue} % CV color - options include: 'blue' (default), 'orange', 'green', 'red', 'purple', 'grey' and 'black'

\usepackage{color}

\usepackage{lipsum} % Used for inserting dummy 'Lorem ipsum' text into the template

\usepackage[scale=0.90]{geometry} % Reduce document margins
\setlength{\hintscolumnwidth}{4cm} % Uncomment to change the width of the dates column



\firstname{Kirill} % Your first name
\familyname{Mokrov} % Your last name

\title{Curriculum Vitae}


%----------------------------------------------------------------------------------------

\begin{document}

\textit{\Huge{\textcolor{gray}{Mokrov Kirill}}}

\hrulefill

\section{Personal Information}
  \cvitem{Email}{kikke88@yandex.ru}
  \cvitem{GitHub}{kikke88}

\section{Education}
\cvitem{}{
    \textbf{Moscow State University}\newline 
    Faculty of Computational Mathematics and Cybernetics\newline 
    Department of Supercomputers and Quantum Informatics
    \begin{itemize}
        \item Bachelor's degree, 2016 -- 2020, GPA 4.75 / 5.0
        \item Master's degree, 2020 -- until now
    \end{itemize}
    }
\section{Technical skills}
    \cvitem{General}{Data structures, Algorithms, Object-oriented programming,\newline Basic knowledge of the Linux/Unix operating system}
    \cvitem{Languages}{C++, Python}
    \cvitem{Libraries}{+: MPI, OpenMP, Numpy\newline +-: CUDA, PAPI, POSIX Threads, FFTW}
    \cvitem{Technologies}{Git, Redis, RabbitMQ}

\section{Strengths}
    \cvitem{}{Hard-working, Communication, English (Intermediate)}
    \cvitem{}{Time Management, Critical Thinking}


\section{Computer practicum projects} {
    \small
	\cvlistitem{\textbf{Finite fields and BCH codes} - Basic operations for working with polynomials in $F_2^q$. Systematic coding procedure for cyclic code defined by its generating polynomial. Procedure of decoding the BCH code using the PGZ method and method based on the extended Euclid algorithm.
    }
    \cvlistitem{\textbf{Realization of quantum gates and algorithms} - n-Hadamard, Phase-shift, NOT, CNOT, CPhase-shift gates, Quantum Fourier transform.
    }
    \cvlistitem{\textbf{Syntax analyzer} - Implemented by recursive descendant method. Defining types of all subexpressions. Detects lexical, syntactic and semantic errors.
    }
    \cvlistitem{\textbf{E-store} - Storing information in Redis data structures. Messages are forwarded via RabbitMQ. The buyer can view the goods, add to the cart, see the statistics, in the end get the shopping done.
    }
    \cvlistitem{\textbf{Numerical solution of the equations of change in the magnetic field} - Using an explicit schema in calculations. Calculating a multidimensional Fourier transform using the FFTW library. Implementation of div, rot, and derivate calculations.
    }
    \cvlistitem{\textbf{Parallel implementation of operations with grid data on an unstructured grid } - Mastering basic data structures to represent an unstructured grid, a graph of connections of calculated cells, a portrait of a sparse matrix, multithreaded and multiprocessor parallelization of the simplest operations.
    }
}
\section {Interests and researches}
    \subsection{Graduate qualification work}
    	\cvitem{}{Development of the method for weak scalability predicting of supercomputer applications}
    \subsection{Scientific interests}
    	\cvitem{}{Parallel and High Performance Computing, Quantum computing, GPU computing}
\end{document}
